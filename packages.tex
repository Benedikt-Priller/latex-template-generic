% LTeX: language=en

% Adaptation to national language ----------------------------------------------
\usepackage{babel}
\usepackage{translations}

% Umlaute ----------------------------------------------------------------------
\usepackage[T1]{fontenc}
\usepackage{textcomp}
\usepackage{lastpage}

% font -------------------------------------------------------------------------
\usepackage{lmodern}
\usepackage{relsize}
\usepackage{soul}

% tables -----------------------------------------------------------------------
\PassOptionsToPackage{table}{xcolor}
\usepackage[table]{xcolor}
\usepackage{tabularx}
\usepackage{longtable}
\usepackage{supertabular}
\usepackage{makecell}
\usepackage{array}
\usepackage{ragged2e}
\usepackage{lscape}
\newcolumntype{x}[1]{>{\raggedleft\hspace{0pt}}p{#1}} 
\usepackage{arydshln}

% graphics ---------------------------------------------------------------------
\usepackage[dvips,final]{graphicx}
\usepackage{graphics}
\usepackage{floatflt}
\usepackage{float}
\usepackage{wrapfig}
\usepackage{graphicx}
\usepackage{latex-svg}      % Custom package
\usepackage{latex-drawio}   % Custom package


% other ------------------------------------------------------------------------
\usepackage[titles]{tocloft}
\usepackage{amsmath,amsfonts,amsthm,amssymb,mathtools}
\usepackage{siunitx} 
\usepackage{wasysym}
\usepackage{dsfont}
\usepackage{pgf,tikz}
\usepackage{pgfplots}
\usepackage{enumitem}
\usepackage{xspace}
\usepackage{witharrows}
\usepackage{makeidx}
\usepackage[printonlyused]{acronym}
\usepackage[acronym]{glossaries}
\usepackage{calc}
\usepackage{nicefrac}
\usepackage[version=4]{mhchem}
\usepackage{chemfig}
\usepackage{xcolor}
\usepackage{microtype}
\usepackage{gauss}
\usepackage{nicematrix}
\usepackage{scalerel}

% Simple definition of line spacing and margins etc.
\usepackage{setspace}
\usepackage[
    tmargin=2cm,
    rmargin=1in,
    lmargin=1in,
    margin=0.85in,
    bmargin=2cm,
    footskip=.2in
]{geometry}

% Symbolverzeichnis
\usepackage[intoc]{nomencl}
\let\abbrev\nomenclature
\renewcommand{\nomname}{Abkürzungsverzeichnis}
\setlength{\nomlabelwidth}{.25\hsize}
\renewcommand{\nomlabel}[1]{#1 \dotfill}
\setlength{\nomitemsep}{-\parsep}

\usepackage{varioref}
\usepackage{url}

\usepackage{chngcntr}
\usepackage{ifthen} % bei der Definition eigener Befehle benötigt
\usepackage{todonotes} % definiert u.a. die Befehle \todo und \listoftodos
\usepackage[square,numbers]{natbib} % wichtig für korrekte Zitierweise
\usepackage{vhistory} % Document versioning

% PDF-Optionen -----------------------------------------------------------------
\usepackage{pdfpages}
\usepackage{pdflscape}
\pdfminorversion=5
\usepackage[
    bookmarks,
    bookmarksnumbered,
    bookmarksopen=true,
    bookmarksopenlevel=1,
    colorlinks=true,        % diese Farbdefinitionen zeichnen Links im PDF farblich aus
    anchorcolor=mylink,     % Ankertext
    citecolor=mylink,       % Verweise auf Literaturverzeichniseinträge im Text
    filecolor=mylink,       % Verknüpfungen, die lokale Dateien öffnen
    menucolor=mylink,       % Acrobat-Menüpunkte
    urlcolor=mylink,
    pdftex,
    plainpages=false,       % zur korrekten Erstellung der Bookmarks
    pdfpagelabels=true,     % zur korrekten Erstellung der Bookmarks
    hypertexnames=false,    % zur korrekten Erstellung der Bookmarks
    linkcolor=black,
    linktoc=all,
]{hyperref}
\hypersetup{
    pdftitle={\titel -- \untertitel},
    pdfauthor={\autorName},
    pdfcreator={\autorName},
    pdfsubject={\titel -- \untertitel},
    pdfkeywords={\titel -- \untertitel},
}
%\usepackage{eforms}

%\usepackage{sectsty}   % For customizing section style
\usetikzlibrary{angles,
                arrows,
                arrows.meta,
                calc,
                decorations.pathreplacing,
                external,
                fit,
                fpu,
                mindmap,
                patterns,
                patterns.meta,
                positioning,
                quotes,
                shadows,
                shapes,
                shapes.geometric,
                trees
                }
\usepackage{venndiagram}
\usepackage{lipsum} 

\usepackage[varbb]{newpxmath}
\usepackage{xfrac}
\usepackage[makeroom]{cancel}
\usepackage{bookmark}
\usepackage{booktabs}
\usepackage{theoremref}
\usepackage[most,many,breakable]{tcolorbox}
\usepackage{varwidth}
\usepackage{etoolbox}
%\usepackage{authblk}
\usepackage{nameref}
\usepackage{multicol}
\usepackage{multirow}
\usepackage{tikz-cd}
\usepackage[ruled,vlined,linesnumbered]{algorithm2e}
\usepackage{comment} % enables the use of multi-line comments (\ifx \fi) 
\usepackage{import}
\usepackage{xifthen}
\usepackage{transparent}
\usepackage{ulem}

\newcommand\mycommfont[1]{\footnotesize\ttfamily\textcolor{blue}{#1}}
\SetCommentSty{mycommfont}
\newcommand{\incfig}[1]{%
    \def\svgwidth{\columnwidth}
    \import{./figures/}{#1.pdf_tex}
}

\usepackage{tikzsymbols}
\renewcommand\qedsymbol{$\Laughey$}